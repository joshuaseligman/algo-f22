%%%%%%%%%%%%%%%%%%%%%%%%%%%%%%%%%%%%%%%%%
%
% CMPT 435
% Fall 2022
% Assignment 3
%
%%%%%%%%%%%%%%%%%%%%%%%%%%%%%%%%%%%%%%%%%

%%%%%%%%%%%%%%%%%%%%%%%%%%%%%%%%%%%%%%%%%
% Short Sectioned Assignment
% LaTeX Template
% Version 1.0 (5/5/12)
%
% This template has been downloaded from: http://www.LaTeXTemplates.com
% Original author: % Frits Wenneker (http://www.howtotex.com)
% License: CC BY-NC-SA 3.0 (http://creativecommons.org/licenses/by-nc-sa/3.0/)
% Modified by Alan G. Labouseur  - alan@labouseur.com
%
%%%%%%%%%%%%%%%%%%%%%%%%%%%%%%%%%%%%%%%%%

%----------------------------------------------------------------------------------------
%	PACKAGES AND OTHER DOCUMENT CONFIGURATIONS
%----------------------------------------------------------------------------------------

\documentclass[letterpaper, 10pt,DIV=13]{scrartcl} 

\usepackage[T1]{fontenc} % Use 8-bit encoding that has 256 glyphs
\usepackage[english]{babel} % English language/hyphenation
\usepackage{amsmath,amsfonts,amsthm,xfrac} % Math packages
\usepackage{sectsty} % Allows customizing section commands
\usepackage{graphicx}
\usepackage{algorithm, algpseudocode}
\usepackage{listings}
\usepackage{parskip}
\usepackage{lastpage}
\usepackage{color}
\usepackage{qtree}

\allsectionsfont{\normalfont\scshape} % Make all section titles in default font and small caps.

\usepackage{fancyhdr} % Custom headers and footers
\pagestyle{fancyplain} % Makes all pages in the document conform to the custom headers and footers

\fancyhead{} % No page header - if you want one, create it in the same way as the footers below
\fancyfoot[L]{} % Empty left footer
\fancyfoot[C]{} % Empty center footer
\fancyfoot[R]{page \thepage\ of \pageref{LastPage}} % Page numbering for right footer

\renewcommand{\headrulewidth}{0pt} % Remove header underlines
\renewcommand{\footrulewidth}{0pt} % Remove footer underlines
\setlength{\headheight}{13.6pt} % Customize the height of the header

\numberwithin{equation}{section} % Number equations within sections (i.e. 1.1, 1.2, 2.1, 2.2 instead of 1, 2, 3, 4)
\numberwithin{figure}{section} % Number figures within sections (i.e. 1.1, 1.2, 2.1, 2.2 instead of 1, 2, 3, 4)
\numberwithin{table}{section} % Number tables within sections (i.e. 1.1, 1.2, 2.1, 2.2 instead of 1, 2, 3, 4)

\setlength\parindent{0pt} % Removes all indentation from paragraphs.

\binoppenalty=3000
\relpenalty=3000

\algrenewcommand{\algorithmiccomment}[1]{\hskip1em\textit{$//$ #1}}

%----------------------------------------------------------------------------------------
%	TITLE SECTION
%----------------------------------------------------------------------------------------

\newcommand{\horrule}[1]{\rule{\linewidth}{#1}} % Create horizontal rule command with 1 argument of height

\title{	
   \normalfont \normalsize 
   \textsc{CMPT 435 - Fall 2022 - Dr. Labouseur} \\[10pt] % Header stuff.
   \horrule{0.5pt} \\[0.25cm] 	% Top horizontal rule
   \huge Assignment Three  \\     	    % Assignment title
   \horrule{0.5pt} \\[0.25cm] 	% Bottom horizontal rule
}

\author{Josh Seligman \\ \normalsize joshua.seligman1@marist.edu}

\date{\normalsize\today} 	% Today's date.

\begin{document}
\maketitle % Print the title

\section{Linear Search}
\subsection{The Algorithm}\label{linearSearch}
Linear search is a searching algorithm that walks through an array and continues on until either it finds the target element or reaches the end of the array. As shown in Algorithm \ref{algorithm:linearSearch}, the function has to compare the target value with every element in the array until the condition in the while loop becomes false. Since the entire array is being searched, no assumptions have to be made about the initial status of the array, which means that the array does not have to be sorted or in any particular order prior to running the search.

\begin{algorithm}
  \caption{Linear Search Algorithm}
  \label{algorithm:linearSearch}
  %Documentation for algorithmicx: https://texdoc.org/serve/algorithmicx/0
  \begin{algorithmic}[1]
      \Procedure{LinearSearch}{$arr$, $target$}
        \State $i \gets 0$ \Comment{Start at the beginning of the array}
        \While{$i~<~len(arr)~\&\&~arr[i]~!=~target$} \Comment{Search through the entire array or until the target is found}
          \State $i++$
        \EndWhile
        \If{$i == len(arr)$}
          \State $i = -1$ \Comment{Set $i$ to -1 to note that the target is not in the array}
        \EndIf
        \State \Return $i$
      \EndProcedure
  \end{algorithmic}
\end{algorithm}

\subsection{Asymptotic Analysis and Comparisons}
As mentioned in Section \ref{linearSearch}, performing a linear search requires going through each element until the target is found or until the end of the array is reached. Also, worst case, the number of iterations will be equal to the number of elements in the array. Listing \ref{lst:linearSearch} provides the C++ implementation of linear search and demonstrates the need to iterate through the entire list with its loop on lines 6-12. Therefore, as its name implies, linear search runs in linear time $O(n)$.

\section{Binary Search}
\subsection{The Algorithm}\label{binarySearch}
Binary search is a searching algorithm that takes an already sorted list and progressively cuts it in half until there is only 1 element left, which is the one that is being searched for. As shown in Algorithm \ref{algorithm:binarySearch}, each recursive call on lines 8 and 10 moves the start or stop limits to a single side of the midpoint, which effectively cuts the array in half at each level of the recursion tree.

\begin{algorithm}
  \caption{Binary Search Algorithm}
  \label{algorithm:binarySearch}
  %Documentation for algorithmicx: https://texdoc.org/serve/algorithmicx/0
  \begin{algorithmic}[1]
      \Procedure{BinarySerach}{$arr$, $target$, $start$, $stop$}
        \State $out \gets -1$ \Comment{Assume the element is not found, by setting the default output to -1}
        \If{start <= stop} \Comment{Working in a valid range}
          \State $mid \gets \lfloor(start + stop) / 2\rfloor$ \Comment{Get the middle of the range}
          \If{$target == arr[mid]$}
            \State $out \gets mid$ \Comment{Target found at position $mid$}
          \ElsIf{$target < arr[mid]$} \Comment{Target is in bottom half of the array}
            \State $out \gets BinarySearch(arr, target, start, mid - 1)$ \Comment{Do binary search on lower half of array}
          \Else \Comment{Torget is in top half of the array}
            \State $out \gets BinarySearch(arr, target, mid + 1, stop)$ \Comment{Do binary search on top half of array}
          \EndIf
        \EndIf
        \State \Return $out$
      \EndProcedure
  \end{algorithmic}
\end{algorithm}

\subsection{Asymptotic Analysis and Comparisons}

\section{Appendix}
\subsection{Comparisons Table}\label{comparisonsTable}
\begin{center}
  \begin{tabular}{|c|c|c|c|}
    \hline
    Algorithm & List & Comparisons & Time \\
    \hline
    Selection Sort & 666 magic items, shuffled & 221445 & 19916376 ns \\
    \hline
    & 20 Yankees greats, sorted & 190 & 12564 ns \\
    \hline
    & 20 Yankees greats, reversed & 190 & 12104 ns \\
    \hline
    Insertion Sort & 666 magic items, shuffled & 112474 & 8952454 ns \\
    \hline
    & 20 Yankees greats, sorted & 19 & 1586 ns \\
    \hline
    & 20 Yankees greats, reversed & 190 & 13012 ns \\
    \hline
    Merge Sort & 666 magic items, shuffled & 5404 & 1069105 ns \\
    \hline
    & 20 Yankees greats, sorted & 48 & 9518 ns \\
    \hline
    & 20 Yankees greats, reversed & 40 & 6164 ns \\
    \hline
    Quicksort & 666 magic items, shuffled & 8092 & 951497 ns \\
    \hline
    & 20 Yankees greats, sorted & 72 & 8052 ns \\
    \hline
    & 20 Yankees greats, reversed & 75 & 8463 ns \\
    \hline
  \end{tabular}
\captionof{table}{A table of the number of comparisons made and time to complete each sort on a variety of lists.}
\end{center}

\lstset{numbers=left, numberstyle=\tiny, stepnumber=1, numbersep=5pt}

% Colors and lstset for syntax highlighting from https://www.overleaf.com/latex/examples/syntax-highlighting-in-latex-with-the-listings-package/jxnppmxxvsvk
\definecolor{mygreen}{rgb}{0,0.6,0}
\definecolor{mygray}{rgb}{0.5,0.5,0.5}
\definecolor{mymauve}{rgb}{0.58,0,0.82}
\lstset{
  backgroundcolor=\color{white},   % choose the background color
  basicstyle=\footnotesize,        % size of fonts used for the code
  breaklines=true,                 % automatic line breaking only at whitespace
  captionpos=b,                    % sets the caption-position to bottom
  commentstyle=\color{mygreen},    % comment style
  escapeinside={\%*}{*},          % if you want to add LaTeX within your code
  keywordstyle=\color{blue},       % keyword style
  stringstyle=\color{mymauve},     % string literal style
}

\subsection{Linear Search}
\lstinputlisting[caption = Linear Search (C++), label = lst:linearSearch, language = C++, firstline = 6, lastline = 30, firstnumber = 1]{./../searches.cpp}

\end{document}