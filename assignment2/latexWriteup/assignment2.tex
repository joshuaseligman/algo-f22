%%%%%%%%%%%%%%%%%%%%%%%%%%%%%%%%%%%%%%%%%
%
% CMPT 435
% Fall 2022
% Assignment 2
%
%%%%%%%%%%%%%%%%%%%%%%%%%%%%%%%%%%%%%%%%%

%%%%%%%%%%%%%%%%%%%%%%%%%%%%%%%%%%%%%%%%%
% Short Sectioned Assignment
% LaTeX Template
% Version 1.0 (5/5/12)
%
% This template has been downloaded from: http://www.LaTeXTemplates.com
% Original author: % Frits Wenneker (http://www.howtotex.com)
% License: CC BY-NC-SA 3.0 (http://creativecommons.org/licenses/by-nc-sa/3.0/)
% Modified by Alan G. Labouseur  - alan@labouseur.com
%
%%%%%%%%%%%%%%%%%%%%%%%%%%%%%%%%%%%%%%%%%

%----------------------------------------------------------------------------------------
%	PACKAGES AND OTHER DOCUMENT CONFIGURATIONS
%----------------------------------------------------------------------------------------

\documentclass[letterpaper, 10pt,DIV=13]{scrartcl} 

\usepackage[T1]{fontenc} % Use 8-bit encoding that has 256 glyphs
\usepackage[english]{babel} % English language/hyphenation
\usepackage{amsmath,amsfonts,amsthm,xfrac} % Math packages
\usepackage{sectsty} % Allows customizing section commands
\usepackage{graphicx}
\usepackage{algorithm, algpseudocode}
\usepackage{listings}
\usepackage{parskip}
\usepackage{lastpage}
\usepackage{color}

\allsectionsfont{\normalfont\scshape} % Make all section titles in default font and small caps.

\usepackage{fancyhdr} % Custom headers and footers
\pagestyle{fancyplain} % Makes all pages in the document conform to the custom headers and footers

\fancyhead{} % No page header - if you want one, create it in the same way as the footers below
\fancyfoot[L]{} % Empty left footer
\fancyfoot[C]{} % Empty center footer
\fancyfoot[R]{page \thepage\ of \pageref{LastPage}} % Page numbering for right footer

\renewcommand{\headrulewidth}{0pt} % Remove header underlines
\renewcommand{\footrulewidth}{0pt} % Remove footer underlines
\setlength{\headheight}{13.6pt} % Customize the height of the header

\numberwithin{equation}{section} % Number equations within sections (i.e. 1.1, 1.2, 2.1, 2.2 instead of 1, 2, 3, 4)
\numberwithin{figure}{section} % Number figures within sections (i.e. 1.1, 1.2, 2.1, 2.2 instead of 1, 2, 3, 4)
\numberwithin{table}{section} % Number tables within sections (i.e. 1.1, 1.2, 2.1, 2.2 instead of 1, 2, 3, 4)

\setlength\parindent{0pt} % Removes all indentation from paragraphs.

\binoppenalty=3000
\relpenalty=3000

\algrenewcommand{\algorithmiccomment}[1]{\hskip1em\textit{$//$ #1}}

%----------------------------------------------------------------------------------------
%	TITLE SECTION
%----------------------------------------------------------------------------------------

\newcommand{\horrule}[1]{\rule{\linewidth}{#1}} % Create horizontal rule command with 1 argument of height

\title{	
   \normalfont \normalsize 
   \textsc{CMPT 435 - Fall 2022 - Dr. Labouseur} \\[10pt] % Header stuff.
   \horrule{0.5pt} \\[0.25cm] 	% Top horizontal rule
   \huge Assignment Two  \\     	    % Assignment title
   \horrule{0.5pt} \\[0.25cm] 	% Bottom horizontal rule
}

\author{Josh Seligman \\ \normalsize joshua.seligman1@marist.edu}

\date{\normalsize\today} 	% Today's date.

\begin{document}
\maketitle % Print the title

\section{Selection Sort}\label{selectionSortSection}
\subsection{The Algorithm}
Selection sort is a sorting algorithm that, for each iteration of the array, selects the smallest (or largest) element of the unsorted part of the array and places the element into its sorted position. As shown in the pseudocode for the sort in Algorithm \ref{algorithm:selectionSort}, selection sort works with the subset of the array in the range $[i,~n)$ in each iteration because the elements in the indices less than $i$ are already sorted and do not have to be checked. Thus, as more elements get sorted, the quicker each iteration becomes because a smaller portion of the array is compared until $i = n - 2$, which is the final iteration of the algorithm. Selection sort is also very consistent in that it runs in the same amount of time regardless of the order of the elements and has both a best and worst case of $n^2$, which will be analyzed in further detail in Section \ref{selectionAnalysis}.

\begin{algorithm}
  \caption{Selection Sort Algorithm}
  \label{algorithm:selectionSort}
  %Documentation for algorithmicx: https://texdoc.org/serve/algorithmicx/0
  \begin{algorithmic}[1]
      \Procedure{SelectionSort}{$arr$}
        \For{$i \gets 0,~n - 2$} \Comment{Iterate through the second to last element as an array of size 1 is sorted}
          \State $smallestIndex \gets i$
          \For{$j \gets i + 1,~n - 1$} \Comment{Iterate through the remainder of the array}
            \If{$arr[j] < arr[smallestIndex]$}
              \State $smallestIndex \gets j$ \Comment{Set the new smallest index if a smaller element is found}
            \EndIf
          \EndFor
          \State $swap(arr,~i,~smallestIndex)$ \Comment{Place the smallest item in the subarray into its sorted place}
        \EndFor
      \EndProcedure
  \end{algorithmic}
\end{algorithm}

\subsection{Asymptotic Analysis}\label{selectionAnalysis}
Listing \ref{lst:selectionSortListing} contains the C++ code implementing selection sort on lines 6 - 25. Line 6 defines a loop that iterates $n - 1$ times and contains 2 assignments and a comparison, all of which operate in constant time for each iteration. Thus, line 6 will take $(n - 1) * C_{1}$ time, where $C_{1}$ is the time needed for each of the operations. Next, line 8 is an assignment, which take a constant time and executes $n - 1$ times because it is in the outer loop, resulting in a time of $(n - 1) * C_{2}$, where $C_{2}$ is the constant time needed for the assignment. Line 11, similar to line 6, defines a loop with 3 constant time expressions, which can be marked as $C_{3}$. However, since it is nested inside of the loop on line 6, the total number of iterations of the inner loop is more complex. In the first iteration of the outer loop, the inner loop runs $n - 1$ times. From there, each corresponding iteration of the outer loop results in one less iteration of the inner loop with a minimum of 1 pass on the inner loop when $i = n - 2$. Therefore, the total number of times the inner loop on line 11 will be called is $\sum_{k = 1} ^{n - 1} k$, which by the formula for the sum of the first $N$ natural numbers, is equal to $\frac{(n - 1)(n - 1 + 1)}{2} = \frac{1}{2}n^2 - \frac{1}{2}n$. Thus, the total time to execute line 11 is $(\frac{1}{2}n^2 - \frac{1}{2}n) * C_{3}$. Next, line 13 contains a comparison that, since it is nested inside the inner loop, will run in $(\frac{1}{2}n^2 - \frac{1}{2}n) * C_{4}$ time, where $C_{4}$ is the time needed to make the comparison. Line 15 is a simple assignment and, just like line 14, will run in $(\frac{1}{2}n^2 - \frac{1}{2}n) * C_{5}$, where $C_{5}$ is the time to perform the assignment. The assignment on line 18 is purely for collecting data and not part of the algorithm and, therefore, will be excluded from the asymptotic analysis of selection sort. Line 19 is the end of the inner loop, and represents an unconditional branch back to the top of the loop, which means it runs the same number of iterations as the loop, which is $(\frac{1}{2}n^2 - \frac{1}{2}n) * C_{6}$, where $C_{6}$ is the time needed to execute the branch. Next, lines 22-24 are all assigments, which run in constant time, and are located in the outer loop. Thus, they run in $(n - 1) * C_{7}$ time, where $C_{7}$ is the time needed to perform the swap. Lastly, line 25 is the close and unconditional branch for the outer loop, which will run in $(n - 1) * C_{8}$ time, where $C_{8}$ is the time to execute the unconditional branch. Overall, when adding up the runtimes of each line and dropping the constants, the sum is $4 * (n - 1) + 4 * (\frac{1}{2}n^2 - \frac{1}{2}n) = 2n^2 + 2n - 4 \approx n^2 + n$ is $O(n^2)$.

\section{Insertion Sort}\label{insertionSortSection}
\subsection{The Algorithm}
Insertion sort in a sorting algorithm that places an element in its sorted place by sliding previously sorted elements over until the sorted position is found for the element. Unlike selection sort, insertion sort has a unique property in that its performance varies based on the state of the input array. For instance, if the array is already completely sorted, the while loop on line 5 in Algorithm \ref{algorithm:insertionSort} will never be entered because each element is already sorted and in its proper position. This makes the best case runtime of insertion sort $\Omega(n)$ because the inner loop is never run and the outer loop just iteraties through the array once. However, the worst case of insertion sort is when the array is in reverse order. This becomes the worst case because, as shown within the while loop in Algorithm \ref{algorithm:insertionSort}, each element will have to be compared with every other element, which will cause $j$ to end at $-1$ and the element gets inserted at the front of the array. This results in a worst case runtime complexity of $O(n^2)$, which will be analyzed and proved in detail in Section \ref{insertionAnalysis}.

\begin{algorithm}
  \caption{Insertion Sort Algorithm}
  \label{algorithm:insertionSort}
  %Documentation for algorithmicx: https://texdoc.org/serve/algorithmicx/0
  \begin{algorithmic}[1]
      \Procedure{InsertionSort}{$arr$}
        \For{$i \gets 1,~n - 1$} \Comment{Start at index 1 because the first element is already sorted}
          \State $currentVal \gets arr[i]$
          \State $j \gets i - 1$
          \While{$j \geq 0 ~\textbf{and} ~currentVal < arr[j]$} \Comment{Find the position to place the element}
            \State $arr[j + 1] \gets arr[j]$ \Comment{Shift the element over because it is greater than the current value}
            \State $j \gets j - 1$
          \EndWhile
          \State $arr[j + 1] \gets currentVal$ \Comment{Place the element in its sorted position}
        \EndFor
      \EndProcedure
  \end{algorithmic}
\end{algorithm}

\subsection{Asymptotic Analysis}\label{insertionAnalysis}
The code implementation of nsertion sort is found in Listing \ref{lst:insertionSortListing} on lines 7-34. 

\section{Appendix}
\lstset{numbers=left, numberstyle=\tiny, stepnumber=1, numbersep=5pt}

% Colors and lstset for syntax highlighting from https://www.overleaf.com/latex/examples/syntax-highlighting-in-latex-with-the-listings-package/jxnppmxxvsvk
\definecolor{mygreen}{rgb}{0,0.6,0}
\definecolor{mygray}{rgb}{0.5,0.5,0.5}
\definecolor{mymauve}{rgb}{0.58,0,0.82}
\lstset{
  backgroundcolor=\color{white},   % choose the background color
  basicstyle=\footnotesize,        % size of fonts used for the code
  breaklines=true,                 % automatic line breaking only at whitespace
  captionpos=b,                    % sets the caption-position to bottom
  commentstyle=\color{mygreen},    % comment style
  escapeinside={\%*}{*)},          % if you want to add LaTeX within your code
  keywordstyle=\color{blue},       % keyword style
  stringstyle=\color{mymauve},     % string literal style
}

\subsection{Selection Sort}\label{selectionSortListing}
\lstinputlisting[caption = Selection Sort (C++), label = lst:selectionSortListing, language = C++, firstline = 8, lastline = 36, firstnumber = 1]{./../sortsAndShuffles.cpp}

\subsection{Insertion Sort}\label{insertionSortListing}
\lstinputlisting[caption = Insertion Sort (C++), label = lst:insertionSortListing, language = C++, firstline = 38, lastline = 75, firstnumber = 1]{./../sortsAndShuffles.cpp}

\end{document}